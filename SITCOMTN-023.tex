\documentclass[SE,authoryear,toc]{lsstdoc}
%\documentclass{article}

% lsstdoc documentation: https://lsst-texmf.lsst.io/lsstdoc.html
\input{meta}

% Package imports go here.
\usepackage{graphicx}
% Local commands go here.

%If you want glossaries
%\input{aglossary.tex}
%\makeglossaries

\title{SIT-COM Work Management and Organization}

% Optional subtitle
% \setDocSubtitle{A subtitle}

\author{%
Sandrine Thomas, Leanne Guy, Austin Roberts
}

%\setDocRef{SITCOMTN-023}
%\setDocUpstreamLocation{\url{https://github.com/lsst-sitcom/sitcomtn-023}}
%\date{\vcsDate}

% Optional: name of the document's curator
% \setDocCurator{The Curator of this Document}

\setDocAbstract{This management plan covers the organization and management of the work of the System Integration Test and Commissioning (SIT-COM)  team. }

% Change history defined here.
% Order: oldest first.
% Fields: VERSION, DATE, DESCRIPTION, OWNER NAME.
% See LPM-51 for version number policy.
%\setDocChangeRecord{%
  %\addtohist{1}{YYYY-MM-DD}{Unreleased.}{Leanne Guy}
%}


\begin{document}

% Create the title page.
\maketitle
% Frequently for a technote we do not want a title page  uncomment this to remove the title page and changelog.
% use \mkshorttitle to remove the extra pages

% ADD CONTENT HERE
% You can also use the \input command to include several content files.
\section{Introduction}
The goal of this document is to define the process needed for an efficient and timely organization of the work required for a successful commissioning of the Vera C. Rubin Observatory. 
This process includes defining the current priorities along with their coordination both at the SITCom and Project level. 
The main input for priorities sorting are (i) the level 2 and level 3 milestones described in P6, (ii) the managerial work from the SIT-Com leadership team (SCLT) and (iii) the emergent issues.

This document first lays down the different use cases along with their appropriate processes and the prioritization strategy. From these uses cases, we defined the requirements on the Jira tool that will satisfy our needs, including work flow, ticket types, review process, dashboard. Finally, we describe how this Jira project interfaces with the other existing projects. 

\section{Work Management Needs}
\subsection{Processes}
As mentioned in the introduction, there are three main applications for the work management process. 
\subsubsection{Milestone Tracking}
The P6 tool defined two sets of milestones for SITCom, called L2 and L3. Each milestones are reached when the list of associated activities is completed. Both L2 and L3 milestones will be reported on weekly at the SITCom leadership meetings, especially if delays/blockers are encountered, in which case the impact on the schedule/cost/performance might need to be elevated.

\subsubsection{Planned activities not linked to a milestone}
The tickets types and processes are similar to the milestone tracking use case, except that it would start at the Epics level or the Story level, depending on the nature of the task. 

\subsubsection{Emergent Tasks}
Finally, the emergent tasks use case is a mechanism for anyone on the commissioning team to enter issues and bug encountered during commissioning activities. They do not need to be linked to an epic or a milestone while being created. The triage will be done by the SCLT to verify and/or define assignees as well as priorities relative to other work. 

\subsection{Prioritization strategy}
in order to set up the priority accordingly, the activities will need to have a criticality attached to them.  The criticality will be set up by the owner of the ticket and discussed within the SCLT if need be.  We chose to have 5 levels of criticality, as follow:
\begin{itemize}
\item{\bf Critical}: when a tickets is flagged as critical, the assignee(s) are expected to focus only on this particular activity. As a result, each person should have only one critical activity to work on at a given time and managers should adjust the schedule to allow for the other activities to wait, including meetings. 
Usually, the associated tickets (most likely bugs) are blocking critical path activities that are ON-GOING if not done or will result in a severe hampering of night-time operations. This should hardly ever happen and should only last for a short period of time (of the order of a few days). 

\item{\bf High} - A ticket is flagged as high criticality when the activity is needed to be completed in short order and at risk of hitting the critical path if not worked on immediately and with a high amount of dedicated time. Essentially, a single person should only EVER have one of these at a time, potentially combined with other tasks with lower criticality. 

\item{\bf Medium (default)} - This is the standard level and most tasks will be flagged as such. They correspond to off critical path items that have some wiggle room on deadline. Task should be mapped to a milestone, a planned activity or a roadmap (in the case of broader teams).  
Each person should have no more than ~3 of these.

\item{\bf Low} A ticket is flagged as low when it is a nice-to-have not immediately needed. They receive attention when someone has time and  should be worked on a MINIMUM of a few hours per week. They can be elevated in criticality if need be at a later date by the owner of the ticket.  

\item{\bf Undefined} These are the backlogged tickets that no one should be working on.
\end{itemize}

It is important that the SIT-Com managers regularly check the priorities against the overarching project schedule as well as the list of tasks per person to ensure people are not overbooked. 




\section{The SITCom Work management Jira Project }
Considering these high level needs, this section describes the different resulting configurations for the Jira project. 

There will be 5 different ticket types within this Jira project.
There will be two levels of milestones called L2 and L3, linked to the P6 scheduled. Each of the {\bf L2 milestones} requires N number of {\bf L3 milestones} to be accomplished before being reviewed. Each L3 milestone is further decomposed into {\bf Epics} and tracked by one of the SCLT coordinator. 

The epics are linked to {\bf stories}, themselves relating to {\bf tasks}. The stories and tasks are respectively defined by a SIT-Com member and anyone on the project. 
Epics, stories and tasks can be defined outside a milestones when need be.

\subsection{Ticket Type}
The fields required for the different tickets include the title, summary, description, component, assignee, priority. The other fields are Start Date (Non milestones), End Date (non milestones), Labels, Reviewers, Approved by.  The fields specific to each ticket type is listed below.
\begin{itemize}
\item{Milestones}: Milestone level (required), Success Criteria, Due Date, L2 Milestone, L3 Milestones, included Epics
\item{Epics}: Epic name, Sprint, Story points, Supported Milestones, Included Stories/Bugs
\item{Story/Bug}: Epic Link, Sprint, Story Points, Supported Epic, Related Issues, Needs to be done before/after
\item{Tasks}: Story Points, Needs to be dong before/after
\end{itemize}

\subsection{Workflows}
There will be two different workflows, depending on if the ticket type is a milestone or not, as shown in Figure \ref{fig:workflow}. The main differences are the acknowledged and invalid states available in the non-milestone case.  This comes from the nature of the milestones ticket being only an achieved goal instead of an activity that needs to be done. 

\begin{figure}[h]
\begin{center}
\includegraphics[width=0.48\textwidth]{WorkFlowMilestones.png}
\includegraphics[width=0.48\textwidth]{WorkFlowNonMilestones.png}
\caption{\label{fig:workflow} Workflows}
\end{center}
\end{figure}


\subsection{Ticket Creation and Review process}
In order to enforce a common process, the workflow has constraints on who is allowed to create tickets and move them to different state transition. 

Only SIT-COM leads can create, deprecate and/or reopen Milestones tickets. SCLT coordinators and leads can create an Epic. However, only the assignee can invalidate, deprecate, reopen and/or complete the Epic. Stories/bugs are created by a SIT-Com member and the tasks by anyone on the project. 

One particularity of the non-milestone workflow is the required acknowledgement status, only allowed by the assignee, ensuring that the assignee has received the tasks and will act on it. As expected, the assignee as well as the start and end dates are required to transition to the acknowledge state.

Regarding the review process, reviewers must be populated before submitting the ticket for Review as an email will automatically be sent to all reviewers as part of the transition. All reviewers must approve ticket in order for the ticket to transition to Reviewed status, which is done automatically after the approval from the final reviewer. All child and/or included tickets must be approved first. 
In case of disapproval, the system will automatically put the ticket back into In Progress status and only the assignee can send it back to the In Review state. Each reviewer is required to add a comment as part of the review process, independently of the outcome. 
Note that the ticket can not be edited while in In Review, Reviewed state.


Table \ref{tab:owner} summarizes the ownership of the creation and review process of each SIT-Com Jira ticket. 
\begin{table}
\begin{center}
\caption{\label{tab:owner} Summary of Ownership}
\begin{tabular}{|p{1.8cm}|p{1.8cm}|p{1.8cm}|p{1.8cm}|p{1.8cm}|p{1.8cm}}
\hline
Ticket type vs Use Case              & Milestone 2	& Milestone 3& Epics	& Story/Bug	& Tasks \\
\hline
\hline
Milestone Tracking & {\bf Creator}: P6 &{\bf Creator}: P6& {\bf Creator}: One SCLT Coordinator & {\bf Creator}: SIT-Com member & {\bf Creator}: Anyone \\
 & {\bf Reviewer}: SITCom manager or delegate & {\bf Reviewer}: one SCLT lead or manager & {\bf Reviewer}: one SCLT lead & {\bf Reviewer}: chosen by the ticket creator or a SCLT coordinator & {\bf Reviewer}: chosen by the ticket creator or a SCLT coordinator \\
\hline
Planned activities & -- & -- & Same as above & Same as above & Same as above \\
\hline
Emergent issues & -- & -- & -- & Same as above & Same as above \\
\hline
\end{tabular}
\end{center}
\end{table}

\subsection{Link to Other Jira Projects}
This project needs to be integrated with the other already existing Jira projects. The following list covers the potential required links.

\begin{itemize}
\item Summit: this is the Jira project used to define the activities being conducted at the summit facilities everyday. These activities are ranked by priorities 1 to 4.  The SITCom tasks will appear on the summit calendar.
\item DM: the DM project gathered all the activities (planned, bugs, etc) conducted by the software team at large. It includes the pipeline as well as the control software tasks. 
\item IT: the IT project is reserved for all the activities conducted by the IT group, both at the summit, in Tucson and in La Serena. Similarly to the DM project, it covers both planned activities as well as emerging issues. 
\item CAP: CAP stands for the Commissioning Activities Planning and is used for work coordination of software needed for commissioning. There was some discussion to extend this particular Jira project to match our needs, however, there was enough differences that we decided not to go that route. As a result, this project might be merge into the SITCom project as we move forward.  
\item LVV: LSST Verification and Validation project gathers all the requirement verification and validation procedures as well as incremental testing procedures.
\item Obs: the Observing Operation Jira project was created for night observation mostly, during which issues are encountered and need to be addressed within the next day or two. Its workflow is basic, allowing for quick reporting, requiring the observer to only add essential details (fault title and description, time of the fault, urgency). The triage and redirection to appropriate assignees and Jira projects is being done by a delegate of each technical group as soon as possible the next day.
\item FRACAS: Failure Reporting, Analysis and Corrective Action System. 
\item Risk: Risk management which we probably will not need to link to. 
\end{itemize}


\appendix
% Include all the relevant bib files.
% https://lsst-texmf.lsst.io/lsstdoc.html#bibliographies
\section{References} \label{sec:bib}
%\renewcommand{\refname}{} % Suppress default Bibliography section
%\bibliography{local,lsst,lsst-dm,refs_ads,refs,books}

% Make sure lsst-texmf/bin/generateAcronyms.py is in your path
\section{Acronyms} \label{sec:acronyms}
\input{acronyms.tex}
% If you want glossary uncomment below -- comment out the two lines above
%\printglossaries





\end{document}
